
\chapter{Reference Model Construction}
%%%%%%%%%%
\begin{itemize}
	\item Approach from Case advanced to the model itself
	\item ECIS model schemata
	\item type / instance : hybride leistungsbündel
\end{itemize}
\subsubsection{A proposed Architecture for Reference Models in BPO}
	%%%%%%%%%%
	deloitte W14: managing change dispute:::: innovation und so... wichtig für begründung des frameworks
	\section{Process Framework}
	%%%%%%%%%%
	\begin{itemize}
		\item meise 2001
	\end{itemize}
	%%%%%%%%%%
	\section{Internal Services}
	%%%%%%%%%%
	%%%%%%%%%%
	\subsection{...}
	%%%%%%%%%%
	%%%%%%%%%%
	\section{Client Services}
	%%%%%%%%%%
	%%%%%%%%%%
	\subsection{...}
	%%%%%%%%%%
	%%%%%%%%%%
	\section{Customer Services}
	This analysis is necessary due to the complex and highly variable processes in Customer Service. Process Identification for Customer Service in the field of the After Sales Service as a Basis for “Lean After Sales Service”
	
	im hippner buch it automation chapter für self service!
	%%%%%%%%%%
	Self Service: Servitization paper 1988!
	%%%%%%%%%%
	\subsection{...}
	%%%%%%%%%%
	%%%%%%%%%%
