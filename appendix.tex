\section{Customer relationship management}\label{sec:appendix01}

\subsection{CRM Notions}
\label{app:crm}
brenneckes crm defs und so 
Appendices provide only two structural levels, \viz, \texttt{\textbackslash section}, and \texttt{\textbackslash subsection}.

Search on scopus, queries are column headings searched in title, abstract and keywords. 

\begin{table}[caption={CRM Publication Comparison}, label=tab:crmnotioncomparison]
	\centering

	\begin{tabular}{p{1cm}| p{2cm} |p{4.3cm}|p{3cm}   } 
		\textbf{Year} & \textbf{"CRM"} & \textbf{"Customer relationship management"} & \textbf{"Customer Management"} \\ \hline 
		2016          & 211            & 34                                          & 8                              \\
		2015          & 198            & 32                                          & 5                              \\
		2014          & 178            & 35                                          & 4                              \\
		2013          & 193            & 37                                          & 2                              \\
		2012          & 166            & 40                                          & 5                              \\
		2011          & 138            & 44                                          & 7                              \\
		2010          & 131            & 32                                          & 1                              \\
		2009          & 133            & 27                                          & 6                              \\
		2008          & 99             & 22                                          & 2                              \\
		2007          & 118            & 26                                          & 1                              \\
		2006 & 111          & 19                           & 1									 \\
	\end{tabular}
\end{table}

\subsection{Multi- and Omni-Channel Publications}
\label{app:mcoc}
The search is done on scopus and queries are \enquote{TITLE-ABS-KEY ( ( omnichannel  OR  omni-channel )  AND  ( crm  OR  management  OR  retail )  AND  customer )} and \enquote{TITLE-ABS-KEY ( ( multichannel  OR  multi-channel )  AND  ( crm  OR  management  OR  retail )  AND  customer )}, respectively.

\begin{table}[caption={Multi- and Omni-Channel Publication Comparison}, label=tab:crmnotioncomparison]
	\centering
	\begin{tabular}{p{1cm}| p{4cm} |p{4cm}    } 
	\textbf{Year} & \textbf{"Multi-channel"} & \textbf{"Omni-channel"} \\ \hline 
	2016          & 32            & 15                                                                       \\
	2015          & 33            & 10                                                                   \\
	2014          & 30            & 7                                                                   \\
	2013          & 17            & 1                                                           \\
	2012          & 24            & 1                                                              \\
	2011          & 24            &                                                       \\
	2010          & 25            &                                                                \\
	2009          & 34            &                                                       \\
	2008          & 29             &                                             \\
	2007          & 23            &                                                       \\
	2006 & 29         &                           					 \\
\end{tabular}
\end{table}

\subsection{Multi- and Omnichannel Separation}
\label{app:mcoc2}
	Separating multi-channel from omni-channel is difficult, as the latter formed as an amplification of the former. \cite{vorhoef2015retail} try a distinction shown in Table \ref{tab:mcoccomparison}, which is here masked from the retail domain. 
\begin{table}[caption={Multi- and Omni-Channel Comparison}, label={tab:mcoccomparison}]
	\centering
	\begin{tabular}{p{3cm}| p{5cm} |p{5cm}} 
		& \textbf{Multi-channel management}                                   & \textbf{Omni-channel management}                                                              \\ \hline
		\textit{Channel focus}                         & \textit{Interactive channels only}                                    & \textit{Interactive and mass-communication channels}                                                   \\ \hline
		\textit{Channes scope}                                 & \textit{Store, online website and direct marketing}                          & \textit{In addition mobile channels (\ie, smart phone, tablets, apps), social media}                   \\ \hline
		{Separation of channels}                           & Separate channels with no overlap                                  & Integrated channels providing seamless customer experiences                                   \\ \hline
		{Brand versus channel customer relationship focus} & Customer - channel focus                                            & Customer - channel - brand  focus                                                              \\ \hline
		{Channel management}                               & Per channel                                                         & Cross-channel                                                                                 \\ \hline
		{Objectives}                                       & Channel objectives (\ie sales per channel, experience per channel)& Cross channel objectives (\ie, overall customer experience, total sales over channels) \\
		
	\end{tabular}
	\quelle{adapted from \citep[\p{176}]{vorhoef2015retail}}
\end{table}
Aspects in channel focus and scope can be criticized in this juxtaposition. It is questionable that channels are excluded from multi-channel management, because in essence the distinction is seen in the relationship \textit{between} channels and not the channels themselves. The excluded channels from multi-channel management (\viz, mass communication) go back to the channel definition of \cite{Neslin2006}, which emphasizes interaction between customer and company. From this, \citeauthor{vorhoef2015retail} infer that solely two-way communication channels can be part of multi-channel management. The understanding in this work is different and makes no difference in possible channel focus and scope among omni- and multi-channel management. As \citeauthor{vorhoef2015retail} describe multi- and omni-channel as \enquote{phases}, they put mobile channels and social media as additions from multi- to omni-channel. This view might be reasoned in the publication time, because multi-channel publications in the early 2000s could not predict the impact of smart phones and tablets on marketing, as the more recent omni-channel publications. Appendix \ref{app:mcoc} holds an overview about publications over time regarding omni- and multi-channel management and proves the greater impact of multi-channel management over omni-channel management in the literature. 


\subsection{Outsourcing Provider Processes}
\label{app:provproc}

		\begin{figure}[caption={Outsourcing Provider Processes}, label={fig:scheweproc}]
	{	\includegraphics[width=.8\textwidth]{figures/scheweproc.pdf}\\
		\quelle{\citep[\p{98}]{schewe2007}} } 
\end{figure}
\subsection{Selected Reference Models}

\label{app:refmods}
	\begin{figure}[caption={SCOR Model}, label={fig:scor}]
	{	\includegraphics[width=.8\textwidth]{figures/scor.pdf}\\
	\parbox{.8\textwidth}{\quelle{\citep{APICS2015}}}} 
\end{figure}

	\begin{figure}[caption={Retail-H}, label={fig:retailh}]
	{	\includegraphics[width=.6\textwidth]{figures/retailh.pdf}
	\\ \parbox{0.6\textwidth}{\quelle{\citep{becker2004handelsinformationssysteme}}}}

	
\end{figure}

\subsection{icebricks Example}
\label{app:iceb}
\begin{figure}[caption={icebricks Process Structure Example: Retail-H \acrshort{CRM} Process}, label={fig:soldes}]
	\begin{subfigure}[c]{.32\textwidth}
		\begin{tikzpicture}
		[node distance=.5cm, start chain=going below,font=\sffamily]
		\node[punktchain, rounded corners=0pt, fill= gray!10, join=by {-}] (eins)      {maintain customer master data};
		\node[punktchain, rounded corners=0pt, join=by {-}] (zwei)      {maintain customer contacts};
		\node[punktchain, rounded corners=0pt, join=by {-}] (drei)      {perform goods planning};
		\node[punktchain, rounded corners=0pt, join=by {-}] (drei)      {...};
		\end{tikzpicture}
		\caption{Selection of CRM Process Components (Detail Processes)}\label{fig:retailh:main}
	\end{subfigure}
\begin{subfigure}[c]{.05\textwidth}
	\begin{tikzpicture}
	\end{tikzpicture}
\end{subfigure}
	\begin{subfigure}[c]{.45\textwidth}
		\begin{tikzpicture}
		[node distance=.5cm, start chain=going below,font=\sffamily]
		\node[punktchain, join=by {-}] (eins)      {maintain customer basic data};
		\node[punktchain,join=by {-}] (zwei)      {maintain customer views};
		\node[punktchain,  join=by {-}] (drei)      {evaluate customer roles};
		\node[punktchain,  join=by {-}] (drei)      {...};
		\end{tikzpicture}
		\caption{Selection of Maintain Customer Master Data: Process Building Blocks}\label{fig:retailh:detail}
	\end{subfigure}
	
\end{figure}

\subsection{Knowledge Management frameworks}
\label{app:knowmang}

\begin{table}[caption={Knowledge Management Framework Options}, label=tab:knowmangoptions]
	\centering
	
	\begin{tabular}{p{1cm}| p{2cm} |p{2cm}|p{3cm} | p{5cm  } }
		\textbf{Year} & \textbf{Author} & \textbf{Type} & \textbf{Decision} & \textbf{Activities} \\ \hline 
		2000          & Andersen Consulting            &      Technical Report                                      & \text{\sffamily X} URL not found         & Acquire, Create, Synthesize, Share, Use to Achieve                     \\
		1999          & Knowledge Associates            & Technical Report                                       &  \text{\sffamily X} URL not found & Acquire, Develop, Retain, Share                             \\
			1997          & Van Heijst \etal            & Paper                                          &  \checkmark & Development, Consolidation, Distribution, Combination                              \\
		1996          & Marquardt            & Book                                          &  \text{\sffamily X} Book not available & Acquisition, Creation, Transfer \& Utilization, Storage                             \\
	


	\end{tabular}
	\quelle{adapted from \citep[\pf{8}]{Rubenstein_Montano_2001}}
\end{table}
\subsection{Financial Engineering in the Outsourcing Deal}
\label{app:fineng}
	\begin{figure}[caption={Financial Engineering in the Outsourcing Deal}, label={fig:scheweproc}]
	{	\includegraphics[width=.8\textwidth]{figures/financialengineering.pdf}
		
 }
 \parbox{.6\textwidth}{\quelle{\citep[\p{32}]{deloittehandbook}}}

\end{figure}


\subsection{Frameworks for the Product Development process}
\label{app:pdframeworks}
% Please add the following required packages to your document preamble:
% \usepackage{multirow}
\begin{table}[caption={Product Development Process Derivation}, label=tab:nsdframeworkds]
	\centering
\begin{tabular}{p{4cm}|p{5cm} |p{4.7cm}}
	\textbf{\cite{cowell1988new}} & \textbf{\cite{Edgett_1996}} adapted from \cite{cooper1988new}  & \textbf{Process in thesis}   \\ \hline \hline
	idea generation                 &                                         \\ \hline
	idea screening                  & idea screening   & evaluate idea                      \\ \hline
	& \textbullet \: preliminary market assessment     &assess market requirements    \\
	& \textbullet \: preliminary technical assessment  &assess technical requirements       \\
	& \textbullet \: detailed market study / market research \\ \hline
	concept development and testing &                       & conceptualize product                  \\ \hline
	business analysis               & business/financial analysis           & perform business analysis  \\ \hline
	\multirow{4}{*}{development}    & \textbullet \: product development      &	\multirow{4}{*}{develop product}               \\ 
	& \textbullet \: process procedures                      \\
	& \textbullet \: system design \& testing                \\
	& \textbullet \: personell training                      \\\hline
	testing                         & test market / trial sell       & test product         \\ \hline
	                      & pre-commercialization \:\:\:\:\:\:\:\:\:\:\:\:\:\:\:\:\:\:\: business analysis \\ \hline
	commercialization               & full-scale launch                   & deliver product    \\ \hline
	& post-launch review \& analysis         
\end{tabular}\\
\quelle{adapted from \citep{Edgett_1996, cowell1988new}}
\end{table}

\subsection{Portfolio Management Definition}
\label{app:pmdef}
\enquote{
Portfolio management is a dynamic decision process, whereby a business's list of active new product (and R\&D) projects is constantly updated and revised. In this process, new projects are evaluated, selected, and prioritized; existing projects may be accelerated, killed, or deprioritized; and resources are allocated and reallocated to the active projects. The portfolio decision process is characterized by uncertain and changing information, dynamic opportunities, multiple goals and strategic considerations, interdependence among projects, and multiple decision-makers and locations.
The portfolio decision process encompasses or overlaps a number of decision-making processes within the business, including periodic reviews of the total portfolio of all projects (looking at the entire set of projects, and comparing all projects against each other), making go/kill decisions on individual projects on an on-going basis (using gates or a stage-gate process), and developing a new product strategy for the business, complete with strategic resource allocation decisions.}  \citep[\p{335}]{cooper1999new}

\subsection{Consultative Selling Process}
\label{app:salesbb}
The use of the \textit{Bid Management} process documentation at Arvato enables modeling of \textsc{Consultative Selling} in the provider model. 

