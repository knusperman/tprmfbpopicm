%!TEX root = ./00_thesis.tex
%!GEDIT texmaster = ./00_thesis.tex

\chapter{Motivation}

Prozessorientierung ist eine nicht mehr wegzudenkende Maxime in der Gestaltung von Unternehmen. Sie ist ein wesentlicher Bestandteil der Forschung in der Be- triebswirtschaftslehre und der Wirtschaftsinformatik.

	\begin{itemize}
		\item Janina BA
		\item ECIS Paper
		\item Refmod motivation Püster?
		\item Omnichannel 
		\item purpose statement
		\item research question and hyptoheses
	\end{itemize}

	\begin{itemize}
		\item Crewsell: State problem, 
		\item review studies that have addressed the problem,
		\item  indicate definciencies in studies, 
		\item advance significance, 
		\item state purpose statement
	\end{itemize}
%%%%%%%%%%
\chapter{Methodology}
%%%%%%%%%%
	\section{Overview}
	%%%%%%%%%%		
		\begin{itemize}
			\item Creswell Preliminary Cosiderations:
			\item 1Selection of Research Approach
			\item epistemology (sarker2013)
			\item 2Litreview
			\item 3UseofTheory ?
			\item 4Writing Strategy?
			\item Figure for my approach
		\end{itemize}
	
	
	\begin{tikzpicture}
	[node distance=.6cm,
	start chain=going below,]
	\node[punktchain, join] (investeringer)      {Investeringsteori};
	\node[punktchain, join] (perfekt) {Det perfekte kapitalmarked};
	\node[punktchain, join, ] (emperi) {Emperi};
	\node (asym) [punktchain ]  {Asymmetrisk information};
	\begin{scope}[start branch=venstre,
	%We need to redefine the join-style to have the -> turn out right
	every join/.style={->, thick, shorten <=1pt}, ]
	\node[punktchain, on chain=going left, join=by {<-}]
	(risiko) {Risiko og gamble};
	\end{scope}
	\begin{scope}[start branch=hoejre,]
	\node (finans) [punktchain, on chain=going right] {Det finansielle system};
	\end{scope}
	\node[punktchain, join,] (disk) {Det imperfekte finansielle marked};
	\node[punktchain, join,] (makro) {Investeringsmssige konsekvenser};
		\node[punktchain, draw=pink] (aux1) { };
	\node[punktchain, below right=0.6cm and -1.975cm of makro] (konk) {Konklusion};
	\node[punktchain,  left = of konk,] (konk2) {Konklusi2on};
		\node[punktchain, below= of aux1 ] (m1) {m1};
	% Now that we have finished the main figure let us add some "after-drawings"
	%% First, let us connect (finans) with (disk). We want it to have
	%% square corners.
	\draw[|-,-|,->, thick,] (finans.south) |-+(0,-1em)-| (disk.north);
	\draw[|-,-|,->, thick,] (emperi.south) |-+(0,-1em)-| (risiko.north);
	\draw[|-,-|,->, thick,] (emperi.south) |-+(0,-1em)-| (asym.north);

\draw[|-,-|,->, thick,] (makro.south) |-+(0,-0.7em)-| (konk.north);
\draw[|-,-|,->, thick,] (makro.south) |-+(0,-0.7em)-| (konk2.north);

\draw[|-,-|,->, thick,] (konk.south) |-+(0,-0.7em)-| (m1.north);
\draw[|-,-|,->, thick,] (konk2.south) |-+(0,-0.7em)-| (m1.north);

	% Now, let us add some braches. 

	\draw[tuborg, decoration={brace}] let \p1=(perfekt.north), \p2=(emperi.south) in
	($(2.5, \y1)$) -- ($(2.5, \y2)$) node[tubnode] {Problemfelt};
	\end{tikzpicture}
		\begin{itemize}
			\item Designing research Creswell
			
		\end{itemize}
	%%%%%%%%%%
	\section{Literature Review}
	%%%%%%%%%%
		\begin{itemize}
			\item Where, How, For what
			\item später: was wurde konkret durchsucht und wo? 
		\end{itemize}
	%%%%%%%%%%
	\section{Empirical Research}
	%%%%%%%%%%
		\begin{itemize}
			\item Qualitative Methods
			\item Interviews, how many, how long, with whom, sample
			
		\end{itemize}
	%%%%%%%%%%
	\section{Design Science}
		\begin{itemize}
			\item Hefner, Peffers. Draw line complete the view of used research methods.
		\end{itemize}
%%%%%%%%%%
\chapter{Research Background}
	%%%%%%%%%%
	DEFINITIONS DEFINITIONS DEFINITIONS
	\section{Domain}
	%%%%%%%%%%
		%%%%%%%%%%
		\subsection{Business Process Outsourcing}
		%%%%%%%%%%
		\begin{itemize}
			\item DEF
			\item schewe 3: venn diagramm prozesoptimierung - bpo - outsourcing
			\item IT as an enabler
			\item offshore, nearshore, inshore - nur nennen und auf interessante Aspekte für case eingehen
			\item Model ? Matrix ? 
			\item 
		\end{itemize}
		\subsubsection{Parent Company}
		\begin{itemize}
			\item reasons focus on core competencies
			\item fields // processes no29 for India, Deloitte Outsourcing Paper
			\item challenges
			\item 
		\end{itemize}
		\subsubsection{Outsourcing Provider}
		\begin{itemize}
			\item dienstleistungsstruktur  : schewe
			\item 
			\item no29: intro to firms, transaction costs so on
			\item no29: capabilities of Providers
		\end{itemize}
		%%%%%%%%%%
		\subsection{Customer Management}
		%%%%%%%%%%
		\begin{itemize}
			\item CSM or CM?
			\item Value Chain
			\item Importance for businesses
		\end{itemize}
		%%%%%%%%%%
		\subsection{Customer Relationship Management Business Process Outsourcing Providers}
		%%%%%%%%%%
		\begin{itemize}
			\item ECIS Paper...
			\item Strategy, Capabilities from voleti concretized. 
			\item one face to the customer
			\item B2B2X
			\item virtual company
		\end{itemize}
	%%%%%%%%%%
	\section{Reference Modeling}
	%%%%%%%%%%
		%%%%%%%%%%
		\subsection{Concept}
		%%%%%%%%%%
		\begin{itemize}
			\item general concept
			\item grafik von enzyklopaedie der wi ?
		\end{itemize}
		%%%%%%%%%%
		\subsection{Benefits}
		%%%%%%%%%%
		\begin{itemize}
			\item  ECIS paper part!
			\item TODO: Fragen ob man hier schon auf bpo crm aspekte eingeht... eigentlich schon. was bringen die allgemeinen denn. 
		\end{itemize}
		%%%%%%%%%%
		\subsection{icebricks as a means}
		%%%%%%%%%%
		\begin{itemize}
			\item Icebricks paper von design science conf
			\item Retail Tabelle mit Features aufdröseln 
			\item Warum so Referenzmodellierung machen? Noch nich auf case gehen, das ist background
		\end{itemize}
\chapter{Case}
%%%%%%%%%%
\begin{itemize}
	\item Arvato talk in general
	\item numbers
	\item history
\end{itemize}
%%%%%%%%%%
\section{DSR Application for Arvato}
%%%%%%%%%%
\begin{itemize}
	\item wie wende ich DSR hier an?
	\item warum wende ich DSR hier an?
	
\end{itemize}
%%%%%%%%%%
\section{Problem Identification}
%%%%%%%%%%
\begin{itemize}
	\item Kraume interview, process interviews
	\item historisch gewachsen
	\item verschiedener sprech
	
\end{itemize}
%%%%%%%%%%
\section{Solution Objective and Stakeholders}
%%%%%%%%%%
\begin{itemize}
	\item Kraume interview
	\item Stakeholders
	\subitem Sales
	\subitem PSD / IT
	\subitem Ops
\end{itemize}
%%%%%%%%%%
\section{Limitiations}
%%%%%%%%%%
\begin{itemize}
	\item only core processes specified, as also done in retail-h
	\item core processes partly on lowest level
	\item process refmod, not data
	\item specified later
\end{itemize}
%%%%%%%%%%

\chapter{Reference Model Construction}
%%%%%%%%%%
	%%%%%%%%%%
	\section{Process Framework}
	%%%%%%%%%%
	\begin{itemize}
		\item meise 2001
	\end{itemize}
	%%%%%%%%%%
	\section{Internal Services}
	%%%%%%%%%%
	%%%%%%%%%%
	\subsection{...}
	%%%%%%%%%%
	%%%%%%%%%%
	\section{Client Services}
	%%%%%%%%%%
	%%%%%%%%%%
	\subsection{...}
	%%%%%%%%%%
	%%%%%%%%%%
	\section{Customer Services}
	%%%%%%%%%%
	%%%%%%%%%%
	\subsection{...}
	%%%%%%%%%%
	%%%%%%%%%%
\chapter{Evaluation}
%%%%%%%%%%
	%%%%%%%%%%
	\begin{itemize}
		\item 
	\end{itemize}
	%%%%%%%%%%
	\section{Internal Services}
	%%%%%%%%%%
	%%%%%%%%%%
	\section{Client Services}
	%%%%%%%%%%
	%%%%%%%%%%
	\section{Customer Services}
	%%%%%%%%%%
	%%%%%%%%%%
\chapter{Conclusion}






\chapter{Sample}
This \LaTeX \- template has been developed as an alternative to the well-known Microsoft Word \enquote{Becker-Vorlage}. \path{00_thesis.tex} is the master file.

It is build by  Jan Betzing and Dominik Lekse and draws from the DBIS template by Till Haselmann and Florian Stahl, as well as from the IS template by Stephan Dlugosz.

This document is work-in-progress and provides instructions on how to use the template. It does not give advices on scientific writing.

Please feel free to contribute to this template. Members of the WWU M\"{u}nster can request access to the template by contacting the author at \href{mailto:jan.betzing@ercis.uni-muenster.de}{jan.betzing@ercis.uni-muenster.de}. Afterwards you will be able to clone the template from \path{https://wiwi-gitlab.uni-muenster.de/lsis/isthesis.git}, and create push-requests with their new features.

\paragraph{TODO}
\begin{itemize}
	\item Configuration switch for having \textbackslash chapter\{\} begin on a new page
	\item Replace \texttt{kvoptions} with \texttt{pgfkeys}
\end{itemize}
\section{Elements}
This chapter gives examples on what you can do with this template. It's just a brief overview. Please consult the common sources on how to write sicentific documents and documents with \LaTeX.

\section{Structure}
This template provides three structural levels that appear in the table of contents, \viz, \texttt{\textbackslash chapter}, \texttt{\textbackslash section}, and \texttt{\textbackslash subsection}. Chapters will always start on a new page. Additionally, you can use \texttt{\textbackslash subsubsection} and \texttt{\textbackslash paragraph} as non-hierarchical means to structure your thesis.


\subsection{Lists}
You can use the default \LaTeX \- functions for writing lists, \viz, \texttt{\textbackslash enumerate} for numbered lists and \texttt{\textbackslash itemize} for bullet point lists. Again, the \texttt{\textbackslash subsubsection} and \texttt{\textbackslash paragraph} can be used as structural elements, \eg, when listing definitions of terms.

\subsection{Footnotes}
Footnotes are contiguously numbered throughout the whole document. Use the \texttt{\textbackslash footnote\{text\}} command.  They appear on the page their reference is on \footnote{This is an exemplary footnote.}. Footnotes have to be placed without whitespace behind the word and within the sentence boundaries, \ie, before the period.

\subsection{ToDo-Notes}
You can use ToDo notes using the \texttt{\textbackslash todo\{text\}}  command. Please make sure to remove any ToDo notes before handing in your thesis! \todo[inline]{ToDo: Remove me before publishing}

\section{Formatting Text}
\LaTeX \- provides \texttt{\textbackslash textit\{text\}} for \textit{italics}, \texttt{\textbackslash textbf\{text\}} for \textbf{bold face}, \texttt{\textbackslash texttt\{text\}} for \texttt{typewriter}, \texttt{\textbackslash textsc\{text\}} for \textsc{small caps}, \texttt{\textbackslash underline\{text\}} for \underline{underline}. Additionally, the template provides  \texttt{\textbackslash texthl\{text\}} for \texthl{highlighted text}. Please remove any highlighted text before handing in your thesis!

Please use the \texttt{\textbackslash enquote\{text\}} command for \enquote{direct quotes}.

\subsection{Colors}
This template comes with the colors defined in the \glspl{CD} of the \acrshort{ERCIS} and \acrshort{WWU}. \Tab \ref{tab:colors} lists the color names. You can apply them to text by using the  \\ \texttt{\textbackslash textcolor\{color name\}\{text\}} command.
	
\begin{table}[caption={Colors defined by the template}, label=tab:colors]
	\centering
		\begin{tabular}{@{}ll@{}}
			\toprule
			{\bf Color Name} & {\bf Result} \\ \midrule
			ercis-black      & \textcolor{ercis-black}{Exemplary Text and 0123456789}  \\
			ercis-grey      & \textcolor{ercis-grey}{Exemplary Text and 0123456789}  \\
			ercis-red      & \textcolor{ercis-red}{Exemplary Text and 0123456789}  \\
			ercis-lightred      & \textcolor{ercis-lightred}{Exemplary Text and 0123456789}  \\
			ercis-blue      & \textcolor{ercis-blue}{Exemplary Text and 0123456789}  \\
			ercis-darkblue      & \textcolor{ercis-darkblue}{Exemplary Text and 0123456789}  \\
			ercis-cyan    & \textcolor{ercis-cyan}{Exemplary Text and 0123456789}  \\
			ercis-orange      & \textcolor{ercis-orange}{Exemplary Text and 0123456789}  \\
			ercis-green      & \textcolor{ercis-green}{Exemplary Text and 0123456789}  \\ \midrule
			wwu-black      & \textcolor{wwu-black}{Exemplary Text and 0123456789}  \\
			wwu-green      & \textcolor{wwu-green}{Exemplary Text and 0123456789}  \\
			wwu-lightgreen      & \textcolor{wwu-lightgreen}{Exemplary Text and 0123456789}  \\
			wwu-blue     & \textcolor{wwu-blue}{Exemplary Text and 0123456789}  \\
			wwu-lightblue      & \textcolor{wwu-lightblue}{Exemplary Text and 0123456789}  \\ \bottomrule
		\end{tabular}
\end{table}


\section{Figures}

The \texttt{figure} environment is wrapped around images. These images should either be included as PDF-file via \texttt{\textbackslash includegraphics}, or created via \textit{TikZ/PGF}. For included images, make sure to use high-resolution images, preferably vector images.

Figures float, \ie, they do not necessarily appear at exact the same position you have defined them. Make sure to set a  \textit{caption} and an optional \textit{label} as figure parameters. 

\begin{figure}[caption={Relationship of students and theses}, label={fig:img01}]
	{	\includegraphics[width=.6\textwidth]{figures/figure01.pdf}}
\end{figure}

\subsection{Subfigures}
Sometimes it might be handy to contrast figures, \ie, by placing them next to each other. The template uses the \textit{subcaption} package to provide subfigures. The following example contains two figures, where each subfigure has its own \texttt{\textbackslash label} and \texttt{\textbackslash caption}. Additionally, the whole figure has its own \textit{caption} and \textit{label}. That means, you can reference subfigures  \fig \ref{fig:subfig1} and \fig  \ref{fig:subfig}. Only the whole figure will be listed in the table of figures.

Subfigures are not limited to images, but may also include listings or tables. \Fig \ref{fig:subfig} shows a sample database query expressed in \ac{SQL} (\fig \ref{fig:subfig1}) and as query plan in relational algebra  (\fig \ref{fig:subfig2}).
 
\begin{figure}[caption={Exemplary use of subfigures}, label={fig:subfig}]
	
	\begin{subfigure}[b]{.45\textwidth}
		
		\begin{lstlisting}[nolol, language=SQL]
		SELECT b, d FROM 
			EXAMPLE.RELATION1 r,
			EXAMPLE.RELATION2 s,
		WHERE 
			r.a = 'c'
		AND 
			s.e = 2
		AND 
			r.c = s.c; 
		\end{lstlisting}
		\caption{\gls{SQL} select statement}\label{fig:subfig1}
	\end{subfigure}
	\begin{subfigure}[b]{.53\textwidth}
		\centering	
		\begin{tikzpicture}[node distance = 2cm, auto,
		database/.style={
			cylinder,
			cylinder uses custom fill,
			cylinder body fill=gray!30,
			cylinder end fill=gray!20,
			shape border rotate=90,
			aspect=0.25,
			draw
		}]
		\node [] (queue) {$\Pi_{b, d}$};
		\node [below of=queue] (join) {$\Join_{r.c = s.c}$};
		
		\node [below left of=join,xshift=-1cm] (l1) {$\sigma_{r.a = 'c'}$};
		\node [database, below of=l1] (l2) {\texttt{r}};
		
		\node [below right of=join,xshift=1cm] (r1) {$\sigma_{s.e = 2}$};
		\node [database,below of=r1] (r2) {\texttt{s}};
		
		\draw [<-] (queue) -- (join);
		\draw [<-] (join) -- (r1);
		\draw [<-] (r1) -- (r2);
		\draw [<-] (join) -- (l1);
		\draw [<-] (l1) -- (l2);
		\end{tikzpicture}
		\caption{Sample evaluation plan}\label{fig:subfig2}
	\end{subfigure}
\end{figure}
\section{Listings}
You can use listings to typeset source code. This template uses the \textit{listings} package. Wrap code inside the \texttt{lstlisting} environment and set the \textit{language} (e.g., Java, SQL), \textit{caption}, and optional \textit{label} parameters. If the source code highlighting highlights the wrong keywords or misses keywords, use the \textit{deletekeywords} \resp \textit{morekeywords} parameters. Consult the package documentation for further information.

\begin{lstlisting}[float=htp, caption={Euclid's GCD algorithm implemented in Java}, label={lst:euclid}, language=Java, deletekeywords={}, morekeywords={}]
public class Euclid {

	public static int gcd(int p, int q) {
		if (q == 0) return p;
		else return gcd(q, p % q);
	}

}
\end{lstlisting}

\section{Algorithms}
Some users might require specifying algorithms. This template uses the \textit{algorithm}, \textit{algorithmicx}, and \textit{algopseudocode} packages. Consult the respective manuals for further information. Algorithms do not appear in a table at the beginning of the document, \ie, there is no list of algorithms.

\begin{algorithm}[htb]
	\begin{algorithmic}
		\Require nonnegative integer $a$, nonnegative integer $b$
		\Function{Euclid}{$a, b$}
		\If {$b = 0$} \Comment{comment}
		\State{return $a$;}
		\Else 
		\State {return \textsc{Euclid}$(b, a\mod b)$;}
		\EndIf
		\EndFunction
	\end{algorithmic}
	\caption{Euclid's GCD algorithm in pseudocode}
	\label{alg:garbage}
\end{algorithm}

\section{Acronyms and Abbreviations}
This template provides comprehensive support for acronyms and abbreviations. The template uses the \textit{glossaries} package. 
Please do only define abbreviations and symbols that are uncommon. That means, common abbreviations such as \enquote{\eg} or \enquote{\ie} should not be listed. Abbreviations and symbols are sorted automatically by their label. 

\subsection{Common Abbreviations}
Please note that each full stop in a common abbreviation should be followed by a non-breaking space. This template comes with a variety of macros for common abbreviations, that can be used throughout your theses. The macros differ for English and German theses. Please see the tables below.

\begin{table}[caption={Common abbreviation macros for English theses}, label=tab:macros1]
	\centering
	\begin{tabular}{@{}ll@{}}
		\toprule
		{\bf Command} & {\bf Result} \\ \midrule
			\textbackslash apprx      & \apprx \\
			\textbackslash as      & \as \\
			\textbackslash cf      & \cf \\
			\textbackslash eg      & \eg \\
			\textbackslash Eg      & \Eg \\
			\textbackslash esp      & \esp \\
			\textbackslash etal      & \etal \\
			\textbackslash fig      & \fig \\
			\textbackslash Fig     & \Fig \\
			\textbackslash ie      & \ie \\
			\textbackslash Ie      & \Ie \\
			\textbackslash iid      & \iid \\
			\textbackslash p\{4711\}      & \p{4711} \\
			\textbackslash pf\{4711\}      & \pf{4711} \\
			\textbackslash pp\{11$--$47\}      & \pp{11--47} \\
			\textbackslash resp      & \resp \\
			\textbackslash sect     & \sect \\
			\textbackslash tab      & \tab \\
			\textbackslash Tab      & \Tab \\
			\textbackslash viz      & \viz \\
			\textbackslash wrt      & \wrt \\ \bottomrule
	\end{tabular}
\end{table}

\begin{table}[caption={Common abbreviation macros for German theses}, label=tab:macros2]
	\begin{subfigure}[t]{.45\textwidth}
	\centering
	\begin{tabular}{@{}ll@{}}
		\toprule
		{\bf Command} & {\bf Result} \\ \midrule
\textbackslash aaO & \mbox{a.\,a\,O}\xdot \\
\textbackslash Abb & \mbox{Abb.~} \\
\textbackslash bspw & \mbox{bspw}\xdot \\
\textbackslash bzgl & \mbox{bzgl}\xdot \\
\textbackslash bzw & \mbox{bzw}\xdot \\
\textbackslash ca & \mbox{ca}\xdot \\
\textbackslash dgl & \mbox{dgl}\xdot \\
\textbackslash dsgl & \mbox{dsgl}\xdot \\
\textbackslash dh & \mbox{d.\,h}\xdot \\
\textbackslash etc & \mbox{etc}\xdot \\
\textbackslash eV & \mbox{e.\,V}\xdot \\
\textbackslash evtl & \mbox{evtl}\xdot \\
\textbackslash fs & \mbox{f.\,s}\xdot \\
\textbackslash gdw & \mbox{g.\,d.\,w}\xdot \\
\textbackslash ggf & \mbox{ggf}\xdot \\
\textbackslash hc & \mbox{h.\,c}\xdot \\
\textbackslash iAllg & \mbox{i.\,Allg}\xdot \\
\textbackslash iBa & \mbox{i.\,B.\,a}\xdot \\
\textbackslash idR & \mbox{i.\,d.\,R}\xdot \\
\textbackslash ieS & \mbox{i.\,e.\,S}\xdot \\
\textbackslash inkl & \mbox{inkl}\xdot \\
\textbackslash insb & \mbox{insbes}\xdot \\
\textbackslash Prof & \mbox{Prof}\xdot \\
\textbackslash Dr & \mbox{Dr}\xdot \\
\textbackslash PD & \mbox{PD}\xdot \\
\textbackslash Ing & \mbox{Ing}\xdot \\
\textbackslash iV & \mbox{i.\,V}\xdot \\
\textbackslash iW & \mbox{i.\,W}\xdot \\
\textbackslash iwS & \mbox{i.\,w.\,S}\xdot \\
\textbackslash Nr\{123\} & \mbox{Nr.~123} \\
\textbackslash nW & \mbox{n.\,W}\xdot \\
\textbackslash oa & \mbox{o.\,a}\xdot \\
\textbackslash oAe & \mbox{o.\,\"{A}}\xdot \\
			\textbackslash oae & \mbox{o.\,\"{a}}\xdot \\\bottomrule
\end{tabular}
\end{subfigure}
	\begin{subfigure}[position=t]{.45\textwidth}
		\centering
		\begin{tabular}{@{}ll@{}}
			\toprule
			{\bf Command} & {\bf Result} \\ \midrule

			\textbackslash oE & \mbox{o.\,E}\xdot \\
			\textbackslash oEdA & \mbox{o.\,E.\,d.\,A}\xdot \\
			\textbackslash OEdA & \mbox{O.\,E.\,d.\,A}\xdot \\ 
			\textbackslash oV & \mbox{o.\,V}\xdot \\
			\textbackslash OV & \mbox{O.\,V}\xdot \\
			\textbackslash resp & \mbox{resp}\xdot \\
			\textbackslash S\{123\} & \mbox{S.~123} \\
			\textbackslash Sf\{123\} & \mbox{S.~123~f}\xdot \\
			\textbackslash Sff\{123\} & \mbox{S.~123~ff}\xdot \\
			\textbackslash siehe & \mbox{s.\,o}\xdot \\
			\textbackslash sog & \mbox{sog}\xdot \\
			\textbackslash sS\{123\}  & \mbox{s.\,S.~123}\\
			\textbackslash sSf\{123\} &\mbox{s.\,S.~123~f}\xdot \\
			\textbackslash sSff\{123\}& \mbox{s.\,S.~123~ff}\xdot \\
			\textbackslash stu & \mbox{st.\,u}\xdot \\
			\textbackslash su & \mbox{s.\,u}\xdot \\
			\textbackslash Tab & \mbox{Tab.~} \\
			\textbackslash tw & \mbox{t.\,w}\xdot \\
			\textbackslash ua & \mbox{u.\,a}\xdot \\
			\textbackslash etal & \mbox{et\ al}\xdot \\
			\textbackslash uae & \mbox{u.\,\"{a}}\xdot \\
			\textbackslash uAe & \mbox{u.\,\"{A}}\xdot \\
			\textbackslash uiv & \mbox{u.\,i.\,v}\xdot \\
			\textbackslash usw & \mbox{usw}\xdot \\
			\textbackslash uU & \mbox{u.\,U}\xdot \\
			\textbackslash va & \mbox{v.\,a}\xdot \\
			\textbackslash vgl & \mbox{vgl.~} \\
			\textbackslash Vgl & \mbox{Vgl.~} \\
			\textbackslash vs & \mbox{v.\,s}\xdot \\
			\textbackslash zB & \mbox{z.\,B}\xdot \\
			\textbackslash zT & \mbox{z.\,T}\xdot \\
			\textbackslash zz & \mbox{zz}\xdot \\
			\textbackslash zzgl & \mbox{zzgl}\xdot \\
 & \\ \bottomrule
	\end{tabular}
	\end{subfigure}
\end{table}

\subsection{Custom Abbreviations}
Custom abbreviations are defined in the \path{acronyms.tex} file, using the \\
\texttt{\textbackslash newacronym[longplural=\{<long plural>\}, shortplural=\{<short plural>\}]\\ \{<label>\}\{<short>\}\{<long>\}} command. The \textit{longplural} and \textit{shortplural} parameters are optional. The abbreviations are sorted by their labels. The label is furthermore used to reference the abbreviations in your text. You can do so using commands listed in \tab \ref{tab:glossaries}. In most cases, you just use \textbackslash gls\{<label>\}. On the first occurrence, the full version is displayed, \eg, \acrfull{ERCIS}. Afterwards, the short version will be displayed, \eg, \acrshort{ERCIS}.

You pluralize your abbreviation by adding a \texttt{pl} to the \resp command. This will add a small s to the abbreviation, \eg, \acrshortpl{CD}. \Tab \ref{tab:glossaries} shows custom short and long plural versions of the abbreviation \acrshort{kmu}. You might need this \esp for more complex German abbreviations that do not have a \enquote{s} plural form.

\begin{table}[caption={Commands for printing abbreviations}, label=tab:glossaries]
	\centering
	\begin{tabular}{@{}ll@{}}
		\toprule
		{\bf Command} & {\bf Result} \\ \midrule
		\textbackslash gls\{<label>\}     & \textbackslash acrfull on first occurence, \textbackslash acrshort otherwise \\
		\textbackslash glspl\{<label>\}       &  \textbackslash acrfullpl on first occurence, \textbackslash acrshortpl otherwise \\
		\textbackslash acrshort\{<label>\}       & \acrshort{kmu} \\
		\textbackslash acrshortpl\{<label>\}       & \acrshortpl{kmu} \\
		\textbackslash acrlong\{<label>\}       & \acrlong{kmu} \\
		\textbackslash acrlongpl\{<label>\}      & \acrlongpl{kmu} \\
		\textbackslash acrfull\{<label>\}      & \acrfull{kmu} \\
		\textbackslash  acrfullpl\{<label>\}     & \acrfullpl{kmu} \\ \bottomrule
	\end{tabular}
\end{table}

Only referenced abbreviations will be added to the list of abbreviations.

\subsection{Symbols}
If required, you can define symbols in the \path{symbols.tex} file, using the \\ \texttt{\textbackslash addsymboltolist\{<symbol>\}\{<label>\}\{<name>\}} command. The symbols are sorted by their labels. Please note, regardless of using the symbols in the text, all symbols defined in the symbols file will be output to the list of symbols.

\section{Citations and Bibliography}
This template uses {BibTeX} for bibliographies. It comes with the MISQ style that takes care of proper formating and sorting of your references. Of course, you have to maintain a clean \path{.bib} file that caters all necessary attributes. References will appear in the alphabetical order of the surname of the first author. In case of several works by the same author, they are sorted by year.

Citing in the text is done with the \textbackslash citep[<before>][<after>]\{<citekey>\} command. Citations without parenthesis are done with \textbackslash cite\{<citekey>\}. You can reference authors with \textbackslash citeauthor\{<citekey>\}. However, we suggest typesetting authors in \textsc{Small Caps}, \eg, \textsc{\citeauthor{Hammer2015}} is one father of \ac{BPM}.

\paragraph{Exemplary citations}

\begin{itemize}
	\item \gls{BPM} is an integral management paradigm for building and running effective and efficient organizations  \citep{Hammer2015, VomBrocke2014a}.
	\item A holistic approach to \ac{BPM} goes beyond process modeling and workflow management systems \citep[\p{530}]{VomBrocke2014a}.
	\item See \cite{VomBrocke2014a} for a comprehensive review on \ac{BPM} best practices.
	\item \textsc{\citeauthor{Hammer2015}} lists organizational capabilities for \ac{BPM} \citep[\cf][\pf{9}]{Hammer2015}, while \textsc{vom Brocke} \etal give principles of good \ac{BPM} \citep[\cf][\pp{530--546}]{VomBrocke2014a}.
	\item Two authors are automatically divided by an \enquote{and} in English or an \enquote{und} in German, \eg, \citep{Becker2011}.
	\item \enquote{\ac{BPM} can provide a solid set of capabilities essential to master contemporary and future challenges} \citep[\p{534}]{VomBrocke2014a}.
\end{itemize}

\subsection{Misc}
The name and matriculation number of the student will automatically be displayed on the header of every page when the thesis type \textit{seminar} is selected.

\chapter{Compiling the document}
In order to generate a PDF-file from your \TeX-file you have to run the following commands. We assume you have a master file \path{00_thesis.tex} that you want to typeset.

\begin{lstlisting}[float=htp, caption={Commands to compile this document}, label={lst:compiling}, language=bash, morekeywords={pdflatex, bibtex, makeglossaries}]
pdflatex 00_thesis
pdflatex 00_thesis
makeglossaries 00_thesis
bibtex 00_thesis
pdflatex 00_thesis
pdflatex 00_thesis
\end{lstlisting}

Alternatively, you can use your favorite task runner. This thesis comes with a \textit{Grunt} file to kick-start your \LaTeX writing.

When running, Grunt will monitor your thesis and on file changes, the PDF-file is automatically rebuild using the commands from listing \ref{lst:compiling}.
 
Please make sure to have node.js and the \gls{npm} installed. Now you can open a command prompt at the document root and run the commands in listing \ref{lst:grunt}. 

\begin{lstlisting}[float=htp, caption={Installing and running Grunt}, label={lst:grunt}, language=bash]
# Install Grunt via npm (use sudo on Unix-based OS)
npm install -g grunt-cli

# Install Grunt plugins / dependencies
npm install

# Run the Grunt listener 
grunt
\end{lstlisting}

\section{Known Issues}
Under some configurations on Windows machines, the \texttt{makeglossaries} command silently fails, which results in empty lists of accronyms and symbols. Same goes for the implicitly called \texttt{makeindex} command. In this case, you have to install \texttt{Perl}\footnote{https://www.perl.org/get.html} on your machine.