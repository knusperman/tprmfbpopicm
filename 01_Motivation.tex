%!TEX root = ./00_thesis.tex
%!GEDIT texmaster = ./00_thesis.tex
% !TeX spellcheck = en_US
\chapter{Motivation}

Prozessorientierung ist eine nicht mehr wegzudenkende Maxime in der Gestaltung von Unternehmen. Sie ist ein wesentlicher Bestandteil der Forschung in der Be- triebswirtschaftslehre und der Wirtschaftsinformatik.
As put by Thomas Friedman, "The world is flat". Globalization facilitates combinations value-creating activities in economic networks like never before. The key driver of it is information technology, which sets the base for the connectedness we take for granted today. Its implications on markets and businesses are described in the following section. 

Reference modeling is a central disciple in IS research \cite{Fettke2004, konig1996entwicklung, becker2004handelsinformationssysteme}. It refers to the use and construction of reference models.

vorhoef, lemon!!



Process orientiation is a precept in business organization. It is an essential part of research in business adminstration as well as \acrfull{IS}. To make use of it, models as the language of \acrshort{IS} take an important part. In particular, the reference models support businesses in these reorganization projects. They guide the user and help to incorporate best and common practices so that there is solid foundation to customize the model for the businesse's originalities.  

Outsourcing customer service to external providers is a common practice throughout many industries. Dialling a contact number for a service request often ends up with talking to a service agent anywhere around the world. Several companies have specialized to provide professional customer support using various contact channels. Providing customer relationship management (CRM) as service requires the careful and cost-efficient deployment of contact centres. Such centres are often staffed with hundreds of agents that must be hired and trained before customer contact.
For years, special focus has been put on the voice channel (Loudhouse, 2013). Meanwhile digital trends have affected many areas of life, which implies new challenges in customer relationship management. A recent study revealed that 78.7\% of call centre operations managers point out that their current systems fail to meet future needs, as they are telephone-centric and costs for an architecture overhaul are too high (Dimension Data, 2015). Nowadays consumers can use a plethora of devices and software applications to interact with organizations (Köffer, Ortbach and Niehaves, 2014). As a result, the number of used channels to reach organizations increases. More specifically, analysts have seen a move from the traditional voice channel to digital channels, such as chat or social media. For instance, private instant messengers offer faster and less complicated ways to interact with the company. Digital channels in contact centres now take 42\% of overall interactions and are said to overtake voice by the end of 2016 (Dimension Data, 2016). To this end, multichannel CRM has become a “must-have” for customer service management providers (Agnischock et al., 2015).
In this context, the term omnichannel CRM is increasingly dragging intention. Omnichannel CRM can be distinguished from multichannel CRM by not only providing multiple channels for customer interaction but also through seamless integrations of various channels and their underlying data (Verhoef, Kannan and Inman, 2015), which is a difficult task in CRM. At this point in time, omnichannel CRM is often not realized. However, customers more and more expect that they are able to switch between interaction channels without the loss of information. Contact centre interactions will often require the customer to repeat information again, although he or she has earlier written an email or a chat message to the same company.
Omnichannel CRM also comes with important benefits for organizations. Integrated data throughout various channels allows getting a better understanding of the customer’s profile and wishes through analytical support. Still, 40\% of contact centres have no data analysis tool in place despite of being named the top factor to shape the industry in the next five years (Dimension Data, 2015). 
To this end, organizations can better target marketing campaigns or increase the quality of service provision. To realize this, organizations that use outsourcing need close relations to outsourcing providers, since the integration of channels affects various information systems both at the organization but also at the outsourcing provider. More specifically, CRM business processes need to be harmonized since they often span organizational boundaries.


 

Outsourcing processes have be



	\begin{itemize}
		\item Janina BA
		\item ECIS Paper
		\item Refmod motivation Püster?
		\item Omnichannel 
		\item purpose statement
		\item research question and hyptoheses
	\end{itemize}

	\begin{itemize}
		\item Crewsell: State problem, 
		\item review studies that have addressed the problem,
		\item  indicate definciencies in studies, 
		\item advance significance, 
		\item state purpose statement
	\end{itemize}
%%%%%%%%%%
