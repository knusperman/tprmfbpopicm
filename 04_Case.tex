\chapter{Case}
%%%%%%%%%%
This thesis comes into being as part of the \textit{ERCIS Omni-channel lab powered by Arvato}. This cooperation fosters research in omni-channel CRM through cooperation with Arvato, a leading European outsourcing provider. The focus is set on the CRM services division, which is one of four \textit{solution groups}. While the focus is on the German market, Arvato operates international. The organizational structure can be described as decentralized in the past, but it is intended to integrate the independent country organizations more deeply within the solution group. As clients intensify international outsourcing of customer services to Arvato, a need to deliver an orchestrated outsourcing concept across borders arises. The solution group CRM therefore needs more alignment in their three constituents

\begin{itemize}
	\item Sales \& Business Development,
	\item Portfolio Management \& IT and
	\item Operations.
\end{itemize}

A discussion of organizational structures is not in scope of this work, but this view on the business helps to derive a process structure that is later used in the reference model. As an analogy to the domain of retail, one can see its supply and distribution side in form of Sales \& Business Development and Operations, respectively. The \textit{Sales \& Business Development }organization is oriented towards the outsourcing company, e.g., client. It is the main channel of communication to manage existing and potential clients and hence enables the \textit{supply} of outsourcing contracts and therefore business to the organization.\\
 \textit{Portfolio Management \& IT} organizes available service products and their technological foundation. Especially CRM platforms, their selection and implementation is part of their capabilities. With a decentralized orientation in the past, Arvato faces the problem of a heterogeneous system landscape in client business, as there was no guidance for platform selection. The aspired product orientation at Arvato demands standardization in platforms, so that a managed portfolio becomes necessary. As it is a characteristic of CRM outsourcing that clients dictate parts of the environment, \eg, technology or processes, a BPO provider needs to be flexible to react to these requirements. Interface to Sales \& Business Development are the product portfolio, which is marketed to the client. In addition, it supports in design and instantiation of products for a specific client. An internal view of products constituents, namely people, process, platform, is directed towards implementation of services and their use operational use. 
  \textit{Operations}, on the distribution side in the retail analogy, is oriented towards the customer. It delivers value to the customer. With call center business as core of BPO in CRM, it becomes clear that human resources are one key ingredient of the service delivery.  \\
  
  Drawing from the three described constituents of the Arvato CRM solution group, one can identify three stakeholders in the BPO provider organization. Recalling the BPO Outsourcing chain (\Fig \ref{fig:bpochain}), one part of the provider is linked to the client, another to the customer and the third is located in the center. Applying this logic to the three aforementioned units of Arvato CRM and taking a perspective that is scoped on the essential task of the unit, Sales \& Business Development targets clients and Operations is oriented towards the customer. Distancing from Arvato terminology, one can name these two stakeholders simply \textit{Sales} and \textit{Operations}. \\
  
  Portfolio Management \& IT influences both sides, as well as it acts between the two interfaces. Besides, the central part of the chain can be used to model the stakes of the BPO provider as a whole. With the taken perspective that factors out coordinating activities in the three units, the overall interest in terms of alignment across client businesses and country organizations can be captured in an isolated way. The definition of this stakeholder is necessary, as client or operations act with focus on their objectives within the organizations and put less emphasis on the provider organization as a whole. This third stakeholder is named \textit{Management}. 
  

\section{Use of a Reference Model for BPO providers in CRM}

Applying the aforementioned benefits of reference modeling \ref{sec:03_refmod} to the domain in combination with the three stakeholders, one can map these together as in \Tab \ref{tab:benefitsrf}. 

% Please add the following required packages to your document preamble:
% \usepackage{multirow}
\begin{table}[caption={Benefits of Reference Modelling for BPO-providers in CRM }, label=tab:refmodbpobenefits]
	\centering
	\begin{tabular}{p{1cm} p{2cm} |p{3cm} | p{3cm} | p{3cm} |}
		\cline{3-5}
		&                                     & \multicolumn{3}{c|}{\textbf{Stakeholder}}                                                                                                                                           \\ \cline{3-5} 
		&                                     & \textbf{Sales}                                          & \textbf{Management}                                 & \textbf{Operations}                                                 \\ \hline
		\multicolumn{1}{|l|}{\multirow{2}{*}{\textbf{Designer}}} & Knowledge                           & \multicolumn{3}{p{3cm}|}{\multirow{2}{*}{\parbox[c]{9cm}{Applicable only to researchers, not to stakeholders in the organization}}}                                                                       \\ \cline{2-2}
		\multicolumn{1}{|l|}{}                                   & Economic benefits from applications & \multicolumn{3}{l|}{}                                                                                                                                                               \\ \hline
		\multicolumn{1}{|l|}{\multirow{5}{*}{\textbf{User}}}     & Cost reduction                      & Faster client approach                                  & Reduced coordination effort                         & More efficient processes                                            \\ \cline{2-5} 
		\multicolumn{1}{|l|}{}                                   & Profit aspects                      & Organized preparation of client meetings                & Standardization facilitates better management       & Usage of new concepts leads to improvement of operational processes \\ \cline{2-5} 
		\multicolumn{1}{|l|}{}                                   & Risk reduction                      & \multicolumn{3}{l|}{\parbox[t]{9cm}{Lower risk of incorrect modeling through reference processes}}                                                                                                  \\ \cline{2-5} 
		\multicolumn{1}{|l|}{}                                   & Analysis                            & Customized offering for approached clients              & Organization-wide benchmarking                     & Benchmarking                                                        \\ \cline{2-5} 
		\multicolumn{1}{|l|}{}                                   & Information Exchange                & Structured communication of value proposition to client & Communication of best practices within organizatino & Exchange between client operations                                  \\ \hline
	\end{tabular}
\end{table}
%
The reference model is also driven by the business model of CRM outsourcing providers. Since the outsourcing service is provided for several clients, the provider’s internal organization has to cope with this kind of diversity. Each client has its own contract and different parts of customer service process outsourced. While in general the business objects to work on (e.g., schedule in workforce management) or process steps (e.g., route incoming call) apply to all clients, they will differ on detail level (e.g., Client A will have a different routing logic as Client B and Client C has routing still in- house and outsources only after this process step).
The process differences between distinct client types of CRM outsourcing providers motivate to provide adaptive aspects in the CRM reference model as described in (Delfmann, 2007). In a configurable model the outsourcing provider can configure multiple client models based on the provided services that stay compatible and are linked with the provider model. By doing so, the provider model itself gets a reference character in the organisation. Figure 4 visualizes the model levels of the adaptive reference model. Each model uses the aforementioned four-layer structure.
%
\section{Reference Modeling Approach}
research project. 
arvato. 
decentralized to centralized.
product orientation?
"BPO" 
organizational structure. no repetition of background.
stakeholders

\begin{itemize}
	\item Arvato talk in general ok
	\item servitization
	\item numbers todo
	\item history ok
	\item DSR applied here todo
	\item Refmodel use case in bpo crm todo
	\item vom Brocke grafik begründet mein Vorgehen + DSR
\end{itemize}
%%%%%%%%%%
\section{DSR Application for Arvato}
%%%%%%%%%%
\begin{itemize}
	\item wie wende ich DSR hier an?
	\item warum wende ich DSR hier an?
	
\end{itemize}
%%%%%%%%%%
\section{Problem Identification}
%%%%%%%%%%
\begin{itemize}
	\item Kraume interview, process interviews
	\item historisch gewachsen
	\item verschiedener sprech
	
\end{itemize}
%%%%%%%%%%
\section{Solution Objective and Stakeholders}
%%%%%%%%%%
\begin{itemize}
	\item Kraume interview
	\item Stakeholders
	\subitem Sales
	\subitem PSD / IT
	\subitem Ops
\end{itemize}
%%%%%%%%%%
\section{Limitiations}
%%%%%%%%%%
\begin{itemize}
	\item only core processes specified, as also done in retail-h
	\item core processes partly on lowest level
	\item process refmod, not data
	\item specified later
		\item one company
\end{itemize}
%%%%%%%%%%

