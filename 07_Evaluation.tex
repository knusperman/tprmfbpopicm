
\chapter{Evaluation}

After demonstrating the artifact, its utility, quality and efficacy needs to be evaluated.  The reference model is assessed with respect to the defined solution objectives in \ref{sec:solobj}. Drawing on research in evaluation methodology in \acrshort{DSR} \citep[\p{86}]{Hevner2004}, the following can be attributed to observational and descriptive methods. In terms of observational evaluation, the application at Arvato conforms to a case study. Descriptive evaluation is performed through informed arguments from the knowledge base and scenarios. With respect to solution objectives 1 to 4, the \acrfull{GOM} shall be used. The remaining solution objectives 5 to 7 relate to stakeholders, which are evaluated through scenarios.

Design is iterative, so that during reference model construction evaluation was performed and its feedback was considered. The framework is evaluated in \ref{sec:frdg2}. Each of its processes was in detail assessed by profound theoretical argumentation and reasoning about its fit to the domain reference model. However, the demonstration and evaluation in terms of application to Arvato was performed after construction, so that the case study can be seen as such, \eg, application in a new business environment. 

%%%%%%%%%%

\subsubsection{Solution Objectives 1: Generic Reference Model}

A \enquote{construction of a generic reference model that covers distinguishing processes
	for BPO providers in CRM on concept level} was demanded. The framework and its underlying processes represent the primary outcome of this thesis and make up the reference model. Furthermore, coverage of distinguishing processes for the domain is specified. By focusing on the house's core, the model is scoped towards the most important processes for BPO providers, which relates to the \textit{Relevance} \acrshort{GOM}: By definition a model has to reduce for capturing the essence. Through open interviews with Arvato managers, their view on the business was obtained and used to evaluate the reference model a posteriori through application model creation. It was found that their descriptions fit to the reference processes, so that a covering of distinguishing aspects can be affirmed. 

	Recalling the \textit{Guideline of Correctness}, it is not possible to proof semantical correctness in a reference model. However, the comparison of reference processes derived from literature and their comparison to practice at Arvato substantiates correctness of the model. 
	
As the line of argument for reference model construction is based on theoretic considerations, a generic reference model for BPO providers in CRM is intended. The successful demonstration at Arvato marks the first use case and supports the claim of a generic applicability. However, as only one use case is evaluated within this thesis, the general applicability to BPO providers in CRM needs to be subject of future work. Still, the nature of BPO, \ie, the pooling of various clients in one provider organization, encourages the claim of universality. 

The \textit{Guideline of Economic Efficiency} states that utilization through refinements in the model shall be contrasted by the efforts of modeling. Hence, the question arises what detail level is ideal for the reference model. By utilizing of the icebricks language, this thesis employs a structure that has been used for other reference models \citep{Puster2015, 9783832540920}. 

The \textit{Guideline of Systematic Design}, in addition to the structure given by icebricks, is implemented by referring to process, people and technology components in context of CRM or BPO. Explicitly \textsc{Transition}, \textsc{Solution Design}, \textsc{Product Development}  and \textsc{Portfolio Management} can be named in this regard. \acrshort{ERM}s are used to connect separate process models through common reference structures (\cf \Fig \ref{fig:contacterm}, \Fig \ref{fig:prodstructure}). Lastly, the proposed architecture of a domain reference model, provider model and client models( \cf \Fig \ref{fig:modellevels}) supports structuring of the domain and systematic applications. 

%relevance -- das modeln was wichtig ist DISTINGUISHING
%nterviewees aus verschiedenen bereichen habe ihre sicht auf die dinge dargelegt. model fitting hat gezeigt, dass diese abgedeckt werden. folge: relevant inhalte sind drin. 
%correctness -- semantic. 
%lässt sich nicht beweisen, aber anwendbarkeit auf den fall bei arvato lässt schlüsse zu. future work. 
%economic efficiency -- so lange verfeinert, bis die kosten der verfeinerung größer sind als nutzen.
%systematic design -- beispiel ppt, channels in inb, provider, client model

\subsubsection{Solution Objectives 2: Application for Arvato CRM}

While the reference model intentionally represents BPO providers in CRM, its application at Arvato was part of the research design. The demonstration serves two purposes. On the one hand, from a \acrshort{DSR} perspective, it is a necessary step to show the artifacts utility. On the other hand, from a reference modeling perspective, the creation of an application model is motivated in reference modeling itself. 

To evaluate the satisfaction of the second solution objective, differences between Arvato's business understanding and the conveyed view in the reference model need to be discussed in context of the \textit{Guideline of Correctness}. Arvato has no comprehensive reference model for comparison, which necessitates the examination of mentioned activities and their mapping to the reference model. 

Differences in terminology are occuring inevitably and cause discrepancies. One case is the relationship between \textsc{Operations Management} in the reference model and \textsc{Workforce Management} at Arvato, where semantical overlap lead to a mapping between both notions. 

The coverage of reference processes at Arvato is demonstrated in chapter \ref{chap:demo}. However, the coverage of areas at Arvato, that are not covered in the reference model is of increased interest. Here, the integrating aspect of the  process view is of importance. The existence of a dedicated organizational unit or function at Arvato does not imply an existing process. In this regard, Service Delivery Management (1), (Staff) Performance Management (2), Skill Management (3), Queue Management (4) and Case Management (5) can be named. (1) and (4) are seen as part of \textsc{Operations Management}, (2) and (3) relate to \textsc{Quality Management} or \textsc{People Lifecycle Management}, covering the business or employee view, respectively. (5) is one variant of \textsc{Knowledge Management}. 


\subsubsection{Solution Objectives 3: Documentation of Construction}
One can interpret the guideline of \textit{Clarity} to be addressed by thorough documentation, so that the author's conclusions are reproducible ,\eg, transparent and understandable.  
It is stated that the construction process has to be \enquote{well-documented and building on \acrshort{BPO} and \acrshort{CRM} theory}. Chapter \ref{chap:refmod} covers a comprehensive investigation into each of processes and construction of the framework itself. While \textsc{Inbound} and \textsc{Outbound Service} were reasoned from available practical evidence, all other processes were motivated by a theoretic foundation. This choice is additionally motivated in the employed omni-channel perspective. Literature in this area is rare, especially in customer service, so this can be seen as an  contribution of this work to the knowledge base. 

\subsubsection{Solution Objectives 4: Process Modeling Language }
The process reference model is created using the icebricks language. It fulfills the objective of being a syntactic and semantic formalized modeling language and enables model transfer into other environments. A discussion of icebricks and the \acrshort{GOM}s is given in \ref{sec:iceb}. 

\subsubsection{Solution Objectives 5: Use for Sales Activities}
%The model can be used as a statement of competence for sales activities towards clients.

The following scenario demonstrates the reference model's utility in sales activities. The practical use case needs to consider a provider model.  

Arvato as BPO provider and S, a startup company, are actors in this scenario. S approaches Arvato, because their customer service heavily builds on costly voice interactions and S is growing fast. In addition, S considers launching self-services and a chat functionality on their website. Competencies in this regard are very limited at S and Arvato is invited to present their approach. 

The framework is used by the sales team to visualize the outsourcing procedure in total. S, inexperienced in outsourcing, understands that the outsourcing provider acts as a link between the organization and its customers. In addition, the separated customer processes are noticed that pinpoint to the customer, but do not relate to specific channels at first sight. The sales team argues that in an omni-channel environment, channels should not be used as primary structuring elements, as the customer builds the relationship with S across all channels. S states that CRM systems X was recently implemented, self-service product Y was seen on Arvato's website and S wonders how to use them together. The sales team explains that product Y needs to be adapted to work with X, but it is Arvato's approach to create customized solutions for their clients. S asks how a chat communication is processed. The straight-forward model navigation structure answers the question and makes further detail levels accessible, which can be basis for discussion of individual components. 

The scenario conveyed the use of the artifact in an initial meeting with the client. The model's framework structure, product-orientation and omni-channel customer processes supported comfortable communication with the (inexperienced) prospect and stated Arvato's competence in the CRM and BPO field.

\subsubsection{Solution Objectives 6: Use in Cross-Client Communication}
This solution objective describes the utility of the model in supporting a common understanding across client businesses. As previously stated, the business use requires an application (provider) model.  

The second scenario puts emphasis on internal use of the provider model in a meeting of operations managers. This setting demonstrates use of a provider model as an internal reference model for client businesses. The meeting is an open discussion regarding the use of analytical support in inbound transaction processing. The managers work on different client businesses in different verticals with different channels covered and systems involved. %Some know very detailed process models that are part of used application systems and 

The different client businesses show large diversity, but conform to the provider model, which embodies the required level of abstraction. Starting point of the discussion between participants is located inside \textsc{Inbound Service}. As the detail processes are similar across channels, colleagues are able to exchange ideas despite the fact that manager A only delivers service via mail and manager B is responsible for mail and social. However, both use different CRM systems and are working in different verticals, which renders the use of existing detailed process documentation implausible. Despite these differences, the provider model serves as common ground to exchange best practices. 

This scenario stressed the need inside a BPO organization to have an abstract tool that enables communication despite specifics of client businesses. The artifact is used as a means to provide utility in this regard. 

\subsubsection{Solution Objectives 7: Use in Representing Omni-Channel}

The representation of omni-channel in the reference model is seen in the four customer processes, which are for this case considered as a sub-model, that includes a mapping, reduction and pragmatism property \citep{Stachowiak1973}. Omni-channel is hence the artificial original from which aspects are included in the model. It abstracts and reduces to a process view. The intention is to visualize customer service processes in omni-channel, which conforms to the pragmatism property.  

Complexity of omni-channel environments hinders a discussion from different perspectives. The model is a means to limit and manage complexity to enable communication.

One scenario for use of the omni-channel model is seen in the harmonization of three  stakeholders. The process representation depicts customer service provision in the first place. A channel-independent modeling until the lowest model layer enables a discussion across channels, thereby fostering a consistent customer experience. However, channel-specific process building blocks allow incorporation of peculiarities, such as available data. The logic of case-, transaction- and customer-related information (\cf \Fig \ref{fig:contacterm}) facilitates a structuring of these with respect to CRM systems, from which the model intentionally abstracts. Hence, customer experience, customer service and IT specialists can use the model together. 
